% !TeX root = ../tingtao-book-demo.tex

\chapter{拓扑空间}

\begin{exercise}[date=2009-03] 设函数$y=f(x)$在区间$[-1,3]$上的图形为
  \begin{center}
  \includegraphics[width=6cm]{example-image.pdf}
  \end{center}
  则函数$\displaystyle F(x)=\int_{0}^x f(t)dt$ \paren
  \begin{choices}
    \item 
    \includegraphics[width=6cm]{example-image.pdf}
    \item 
    \includegraphics[width=6cm]{example-image.pdf}
    \item 
    \includegraphics[width=6cm]{example-image.pdf}
    \item 
    \includegraphics[width=6cm]{example-image.pdf}
  \end{choices}
\end{exercise}

\section{拓扑空间的定义}

\begin{exercise} 
  设 $(X, \mathscr{T})$ 为拓扑空间, 其中
  \[\mathscr{T}=\left\{X, \emptyset, A,B\right\},\]
  这里 $A,B$ 为 $X$ 的不同非空真子集.  请问: $A$ 与 $B$ 由什么关系?
\end{exercise}
  
\begin{exercise}
 设 $X,Y$为非空集合, $f: X\to Y$ 为映射, $\mathscr{T}^{\ast}$ 为 $Y$ 上的一个拓扑. 定义
  \[\mathscr{T}=\left\{f^{-1}\left(G\right): G\in \mathscr{T}^{\ast}\right\}.\]
  证明: $\mathscr{T}$ 是 $X$ 上的一个拓扑.
\end{exercise}
  
\begin{exercise}	
  令
  \[\mathscr{T}=\left\{(a,+\infty):a\in\mathbf{R}\right\}\cup\left\{\mathbf{R}, \emptyset\right\}.\]
  证明:  $\mathscr{T}$ 是 $\mathbf{R}$ 上的一个拓扑.
\end{exercise}
  
\begin{exercise}	
  设
  $$\mathscr{T}=\{E_n=\{n, n+1, n+2, \cdots\}: n\in \mathbf{N}_+\}\bigcup\{\emptyset\}$$
  \begin{enumerate}
    \item 请证明$\mathscr{T}$是$\mathbf{N}_+$上的拓扑.
    \item 请写出$(\mathbf{N}_+,\mathscr{T})$中的所有闭集.
  \end{enumerate}
\end{exercise}

\begin{exercise}
 设  $(X, \mathscr{T})$ 为拓扑空间, $a\notin X$. 令
  \[X^{\ast}=X\cup \left\{a\right\},\ \mathscr{T}^{\ast}=\mathscr{T}\cup \left\{X^{\ast}\right\}.\]
  证明: $\mathscr{T}^{\ast}$ 是 $X^{\ast}$ 上的一个拓扑.
\end{exercise}
  
\begin{exercise}	
 设  $(X, \mathscr{T})$ 为拓扑空间, $a\notin X$. 令
  \[X^{\ast}=X\cup \left\{a\right\},\ \mathscr{T}^{\ast}=\left\{\left\{a\right\}\cup U: U\in\mathscr{T}\right\}\cup \left\{\emptyset\right\}.\]
  请问: $\mathscr{T}^{\ast}$ 是 $X^{\ast}$ 上的的拓扑吗 ? 请说明理由.
\end{exercise}
  
\begin{exercise}	
 设 $X$ 为非空集合,
  \[\mathscr{T}=\left\{A: A\subset X,\  {\color{red}\sharp
    (A^c)<\infty}\right\}\cup \left\{\emptyset\right\}.\]
  证明: $\mathscr{T}$ 是 $X$ 上的一个拓扑. 该拓扑称为$X$上的{\color{red}余有限拓扑}.
\end{exercise}
  
\begin{exercise}	
设 $X$ 为非空集合,
  \[\mathscr{T}=\left\{A: A\subset X,\  {\color{red}\sharp
    (A^c)\leq \aleph_0}\right\}\cup \left\{\emptyset\right\}.\]
  证明: $\mathscr{T}$ 是 $X$ 上的一个拓扑. 该拓扑称为$X$上的{\color{red}余可数拓扑}.

\end{exercise}

\begin{exercise} 设  $X$ 为非空集合, $\mathscr{T}_{\gamma}(\gamma\in\Gamma)$ 为$X$上的一族拓扑.  证明: $\displaystyle \bigcap_{\gamma\in\Gamma}\mathscr{T}_{\gamma}$ 也是 $X$ 上的一个拓扑.
\end{exercise}

\begin{exercise}	
 设  $X$ 为非空集合, $\mathscr{T}_1, \mathscr{T}_2$ 为$X$上的两个不同拓扑. 请问: $\mathscr{T}_1\cup\mathscr{T}_2$ 一定也是 $X$ 上的一个拓扑吗? 请说明理由.
\end{exercise}
  
\begin{exercise}	
 设$X=\{a,b,c\}$. 请写出$X$上的所有拓扑.
\end{exercise}

\begin{exercise}
  把$x\to0^{+}$时的无穷小量$\displaystyle\alpha=\int_0^x\cos t^2 \d t, \ \beta=\int_0^{x^2}\tan\sqrt{t} \d t,\ \gamma=\int_0^{\sqrt{x}}\sin t^3 \d t$ 排列起来, 使排在后面的是前一个的高阶无穷小, 则正确的排列次序是 \paren
  \begin{choices}
    \item $\alpha, \beta, \gamma$
    \item $\alpha, \gamma, \beta$
    \item $\beta, \alpha, \gamma$
    \item $\beta, \gamma, \alpha$
  \end{choices}
\end{exercise}

\begin{exercise}
  测试选择题,会根据选项长度自动排版 \paren
  \begin{choices}
    \item 选项选项选项选项选项
    \item 选项2
    \item 选项3
    \item 选项4
  \end{choices}
\end{exercise}

\begin{exercise}
  测试选择题,会根据选项长度自动排版 \paren
  \begin{choices}
    \item 选项选项选项选项选项选项选项选项选项选项
    \item 选项2
    \item 选项3
    \item 选项4
  \end{choices}
\end{exercise}

\begin{exercise}
  当 $x \rightarrow 0^{+}$时, 下列无穷小量中阶最高的是 \paren
  \begin{choices}
    \item $\displaystyle \int_0^x\left(e^{t^2}-1\right) \d t$
    \item $\displaystyle\int_0^x \ln \left(1+\sqrt{t^3}\right) \d t$
    \item $\displaystyle\int_0^{\sin x} \sin t^2 \d t$
    \item $\displaystyle\int_0^{1-\cos x} \sqrt{\sin^3 t} \d t$
  \end{choices}
\end{exercise}



\section{第二个测试节}

\begin{exercise}
  测试填空题,只有下划线 \fillin
\end{exercise}

\begin{exercise}
  测试填空题,有答案,输入为可选参数 \fillin[我是答案]
\end{exercise}



\section{第三个测试节}

\begin{exercise}
  其余的题型没有特殊的命令环境,直接输入即可
\end{exercise}

\begin{exercise}
  设 $A, B$ 皆为非空有界数集, 定义数集
  \[
    A+B=\{z \mid z=x+y, x \in A, y \in B\} .
  \]
  证明:
  \begin{enumerate}[(1)]
    \item $\sup (A+B)=\sup A+\sup B$;
    \item $\inf (A+B)=\inf A+\inf B$.
  \end{enumerate}
\end{exercise}

\begin{proof}
  这是一个证明
\end{proof}

\begin{exercise}
  测试
\end{exercise}

\begin{proof}
  \textbackslash qedhere 命令使得证明结束符号出现在此行末尾,
  一般如果是公式结束的话,需要自行加 \textbackslash qedhere 命令手动调整。
    \[
      f(x) = x^2  \qedhere
    \]
\end{proof}

\begin{exercise}
  测试
\end{exercise}