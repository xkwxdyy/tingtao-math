% !TeX root = ../tingtao-book-demo.tex

\chapter{测试章}

$\mathscr{T}$

\section{第一个测试节}

\begin{exercise}[date=2022-03]
  测试选择题,会根据选项长度自动排版 \paren
  \begin{choices}
    \item 选项1
    \item 选项2
    \item 选项3
    \item 选项4
  \end{choices}
\end{exercise}

\begin{exercise}
  把$x\to0^{+}$时的无穷小量$\displaystyle\alpha=\int_0^x\cos t^2 \d t, \ \beta=\int_0^{x^2}\tan\sqrt{t} \d t,\ \gamma=\int_0^{\sqrt{x}}\sin t^3 \d t$ 排列起来, 使排在后面的是前一个的高阶无穷小, 则正确的排列次序是 \paren
  \begin{choices}
    \item $\alpha, \beta, \gamma$
    \item $\alpha, \gamma, \beta$
    \item $\beta, \alpha, \gamma$
    \item $\beta, \gamma, \alpha$
  \end{choices}
\end{exercise}

\begin{exercise}
  测试选择题,会根据选项长度自动排版 \paren
  \begin{choices}
    \item 选项选项选项选项选项
    \item 选项2
    \item 选项3
    \item 选项4
  \end{choices}
\end{exercise}

\begin{exercise}
  测试选择题,会根据选项长度自动排版 \paren
  \begin{choices}
    \item 选项选项选项选项选项选项选项选项选项选项
    \item 选项2
    \item 选项3
    \item 选项4
  \end{choices}
\end{exercise}

\begin{exercise}
  当 $x \rightarrow 0^{+}$时, 下列无穷小量中阶最高的是 \paren
  \begin{choices}
    \item $\displaystyle \int_0^x\left(e^{t^2}-1\right) \d t$
    \item $\displaystyle\int_0^x \ln \left(1+\sqrt{t^3}\right) \d t$
    \item $\displaystyle\int_0^{\sin x} \sin t^2 \d t$
    \item $\displaystyle\int_0^{1-\cos x} \sqrt{\sin^3 t} \d t$
  \end{choices}
\end{exercise}



\section{第二个测试节}

\begin{exercise}
  测试填空题,只有下划线 \fillin
\end{exercise}

\begin{exercise}
  测试填空题,有答案,输入为可选参数 \fillin[我是答案]
\end{exercise}



\section{第三个测试节}

\begin{exercise}
  其余的题型没有特殊的命令环境,直接输入即可
\end{exercise}

\begin{exercise}
  设 $A, B$ 皆为非空有界数集, 定义数集
  \[
    A+B=\{z \mid z=x+y, x \in A, y \in B\} .
  \]
  证明:
  \begin{enumerate}[(1)]
    \item $\sup (A+B)=\sup A+\sup B$;
    \item $\inf (A+B)=\inf A+\inf B$.
  \end{enumerate}
\end{exercise}

\begin{proof}
  这是一个证明
\end{proof}

\begin{exercise}
  测试
\end{exercise}

\begin{proof}
  \textbackslash qedhere 命令使得证明结束符号出现在此行末尾,
  一般如果是公式结束的话,需要自行加 \textbackslash qedhere 命令手动调整。
    \[
      f(x) = x^2  \qedhere
    \]
\end{proof}

\begin{exercise}
  测试
\end{exercise}