\documentclass{tingtao-homework}

\tingtaosetup{
  exercise = {
    answer-color = black,  % \paren 的答案颜色
    show-paren   = true,   % 显示 \paren 的括号
    show-answer  = true,   % 显示 \paren 答案
    top-sep      = .5em plus .5em minus .2em ,
    bottom-sep   = .5em plus .5em minus .2em ,
    show-date    = true,   % 是否显示日期(统一控制)
  },
  % 选择题 choices 环境的相关参数控制,一般不需要改动
  choices = {
    column-sep  = 1em,    % 选项列之间的最小间隔
    label-pos   = auto,   % 标签的位置
    label-sep   = 0.5em,  % 标签与选项之间的距离
    max-columns = 4,      % 最大列数
  },
}

\title{《点集拓扑》HW1 拓扑空间}
\author{华中师大数统学院 邓勤涛}
\date{\today}



% 自行调用宏包

% 参考文献
% \usepackage[
%   backend = biber,        % 需要使用 biber 编译链或者 latexmk 编译参考文献
%   style   = gb7714-2015   % 格式为国标
% ]{biblatex}



% 自行定义命令环境等



\begin{document}

% 标题
\maketitle


\begin{exercise}
  设 $(X, \mathscr{T})$ 为拓扑空间, 其中
  \[\mathscr{T}=\left\{X, \emptyset, A,B\right\},\]
  这里 $A,B$ 为 $X$ 的不同非空真子集.  请问: $A$ 与 $B$ 由什么关系?
\end{exercise}

\begin{exercise}
  设
  \[
    \mathscr{T}=\{E_n=\{n, n+1, n+2, \cdots\}: n\in \mathbf{N}_+\}\bigcup\{\emptyset\}
  \]
  \begin{enumerate}
    \item 请证明$\mathscr{T}$是$\mathbf{N}_+$上的拓扑.
    \item 请写出$(\mathbf{N}_+,\mathscr{T})$中的所有闭集.
  \end{enumerate}
\end{exercise}

\end{document}