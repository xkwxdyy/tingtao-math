\documentclass{tingtao-beamer}


% 个人信息
\title[实变函数]{第6讲 \quad Lebesgue可测函数}
\author{邓勤涛}
\institute[华中师大]{%
  {\color{red}\url{TingtaoMath@yeah.net}}\\[4mm]
  {\color{purple}\large \kaishu 欢迎访问微师-听涛数学}\\[2mm]
  % {\color{red}\url{https://m.weishi100.com/mweb/classroom/?id=16592}}
  \includegraphics[height=2cm]{weishi-qrcode.jpg}
}
\date{}
\logo{\includegraphics[height=1cm]{tingtao-logo.png}}


% 控制是否在每个 section 和 subsection 前出现一次目录
\TocBeforeSection
\TocBeforeSubsection


\begin{document}


% 封面页
\begin{frame}
  \titlepage
\end{frame}

% 目录
\begin{frame}{目录}
  \tableofcontents
\end{frame}


\begin{frame}{关于每帧的标题}
  标题可以不需要 frametitle,直接用 \{\} 写即可,具体参看源代码
\end{frame}


\section{已经定义好的命令环境}


\begin{frame}{有一些已经定义好的定理类环境}
  beamer 中有一些预定义好的定理类环境,比如定义定理等
  \begin{definition}[测度]
    这是定义
  \end{definition}
  \begin{theorem}[Riesz 表示定理]
    这是定理
  \end{theorem}
  \begin{example}[子环没有单位元]
    这是例子
  \end{example}
\end{frame}

\begin{frame}{有一些已经定义好的定理类环境}
  \begin{proof}
    这是证明
  \end{proof}

  \begin{proof}[证明 $A_n(n\geq 5)$是单群]
    这是证明
  \end{proof}
  
  \begin{fact}
    事实
  \end{fact}

  \begin{corollary}[西罗第一定理的推论]
    推论
  \end{corollary}
\end{frame}

\begin{frame}{自定义定理类环境}
  通过 \textbackslash newtheorem 来定义,具体参看源代码
  \newtheorem{remark}{注}
  \newtheorem{property}{性质}
  \begin{remark}[关于单群的定义]
    单群定义中有“不等于 $\{e\}$”
  \end{remark}
  \begin{property}[可积性]
    测试
  \end{property}
\end{frame}

\begin{frame}{列表环境}
  已经进行了相应的完善,带序号的使用 enumerate 环境,不带序号的用 itemize 环境即可。
  \begin{enumerate}
    \item 第一条
      \begin{itemize}
        \item 第一个
        \item 第二个
      \end{itemize}
    \item 第二条
    \item 第三条
      \begin{enumerate}
        \item 第一条
        \item 第二条
      \end{enumerate}
  \end{enumerate}
\end{frame}

\begin{frame}{“强调”命令和环境}
  重定义了 \textbackslash emph 命令,定义了 emphasize 环境。前者用于内容少的部分的强调,后者用于多内容以及行间公式的强调。两者默认效果均为标红,不加粗。颜色通过可选参数的 color 键值控制,需要加粗只需要在可选参数中输入 bf, textbf, bfseries 中任意一个即可(具体参看源代码)。下面给例子:

  测试 \emph{测试} \emph[color=blue]{测试}

  测试 \emph[bf]{测试} \emph[color=blue,textbf]{测试}
\end{frame}


% 内容
\section{示例}

\subsection{阶梯函数与简单函数}

\begin{frame}{示性函数(Characteristic Function)}
\begin{definition}[示性函数(Characteristic Function)]
  设 $E\subset \mathbb{R}^d$,
    \begin{emphasize}
      \[
        \chi_{E}(x)=
          \begin{cases}
          1,& x\in E,\\
          0,& x\notin E,
          \end{cases}
      \]
    \end{emphasize}
  称为 $E$ 上的\emph[bf]{示性函数}.
\end{definition}
\end{frame}


\begin{frame}{阶梯函数(Step Function)}
\begin{definition}[阶梯函数(Step Function)]
  \emph[textbf]{有限和}
  \begin{emphasize}
    \[f=\sum_{k=1}^Na_k\chi_{R_k}\]
  \end{emphasize}
称为\emph{阶梯函数}, 其中$a_k$为\emph{常数}, $R_k$为\emph{长方形}.
\end{definition}
\end{frame}


\begin{frame}{可测函数的性质}
\begin{block}{性质1}
设 $E\subset\mathbb{R}^d$ 为可测集, 则下列三条等价:
\begin{enumerate}
  \item $f: E\to\mathbb{R}$可测
  \item 对$\mathbb{R}$中 \emph{任何开集} $\mathcal{O}$, $f^{-1}(\mathcal{O})$可测
  \item 对$\mathbb{R}$中 \emph{任何闭集} $F$, $f^{-1}(F)$可测
\end{enumerate}
\end{block}
\end{frame}



\section{参考文献}

\begin{frame}{参考文献}
\begin{thebibliography}{GNN}
  \bibitem{} 北京大学数学系前代数小组编, 王萼芳、石生明\ 修订, \emph{高等代数(第5版)}, 高等教育出版社, 2019.
  %\bibitem{} Walter Rudin, \emph{Principle of Mathematical Analysis}(3rd edition), 机械工业出版社.
  \bibitem{} 樊启斌,  \emph{高等代数典型问题与方法(第1版)},  高等教育出版社, 2021.
  %\bibitem{} W.J. Kaczor and M.T. Nowak, \emph{Problems in Mathematical Analysis I}, Student Mathematical Library Volume 4, 2000.
  %\bibitem{} 徐利治等,  \emph{大学数学解题法诠释},  安徽教育出版社, 1998.
  \bibitem{} 钱吉林, \emph{高等代数题解精粹},  中央民族大学出版社, 2002.
  %\bibitem{} 菲尔金哥尔茨, \emph{微积分学教程}(第二卷,第8版), 高等教育出版社, 2006 年第2版.
  %\bibitem{}  \emph{华东师大数学分析第五版上册},高等教育出版社, 2019.
\end{thebibliography}
\end{frame}

\end{document}